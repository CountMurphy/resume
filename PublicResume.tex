%!TEX TS-program = xelatex
%!TEX encoding = UTF-8 Unicode

% Copyright 2017, Sean Bradly <sb@nsfw.jp>

\documentclass[]{SBResume}


% Personal Info for letterhead
%-------------------------------------------------------------------------------
%	PERSONAL INFORMATION - data common to various documents
%-------------------------------------------------------------------------------

\name{Christian}{Gugas}
\location{Austin, Texas}
\jobtitle{Software Engineer}
\phonenumber{(512) 677-LULZ}
\emailaddr{null@null.nill}
\linkedin{cgugas}
\github{CountMurphy}
\headshot{./images/profile}




%-------------------------------------------------------------------------------
\begin{document}
\makeheader

\makeletterfooter{
  Complete history and references are available upon request. For full resume, up-to-date information, and \textbf{\LaTeX}\ source, see \href{https://github.com/rhythmx/resume}{https://github.com/rhythmx/resume}
}

\begin{resume}
  
  \resumesection{About}
  \begin{resumetext}
    \begin{wrapfigure}{r}{3cm}
    \insertheadshot
    \end{wrapfigure}
    With 15 years of experience in the software engineering and
    computer security fields, I have successfully researched,
    designed, implemented, maintained and delivered many successful
    projects in a wide variety of roles with a very broad spectrum of
    subject matter.\\
    
    Security is a primary focus because it affords an opportunity to
    constantly learn new and interesting things. Everything needs a
    level of security, and using that as an entry point, there will
    always be broad and ever-expanding aspects of technology to
    explore.\\
    
    \textbf{Skills:}
    \begin{resumeitemize2}
    \item{\textbf{General}: Security, R\&D, Reversing, Exploiting, Networking, Performance, Embedded}
    \item{\textbf{Languages}: C, C++, Ruby, Python, Assembly, Shell scripting, Lisp, Java, \LaTeX}
    \item{\textbf{Architectures}: X86, X86\_64, ARM, PIC, OpenRISC, Z80, ATmega }
    \item{\textbf{Software}: Linux, Windows, GCC, Clang, GDB, Emacs, IDA, Autotools, CMake}
    \end{resumeitemize2}
    
    \textbf{Personal Interests:}
    \begin{resumeitemize2}
    \item{\textbf{Music}: Guitars, bass, drums, brass, synths and recently, a thrash metal ukulele}
    \item{\textbf{Home Improvement}: My first house has been keeping me busy with plenty of projects}
    \item{\textbf{Entertaining}: Inviting friends over for smoked meats, craft beers, and fine whisky}
    \item{\textbf{Career}: Attending conferences and local meetups to socialize and stay current}
    \end{resumeitemize2}


  \end{resumetext}


\resumesection{Highlighted Projects}

\resumeentry
    {2014-2015}
    {
      \vspace{0.25cm}
      \begin{tikzpicture}%
        \node[circle, inner sep=0.6cm, fill overzoom image=images/google.png] () {};%
      \end{tikzpicture}
    }
    {Secure Embedded Operating System}
    {Inverse Limit (for Google)}
    {

        As part of Google ATAP's \emph{Project Vault}, Inverse Limit's
        small team of 4 designed and developed a complete computer
        platform with a security focus. All components are open
        source; the board schematics, OS, applications, drivers,
        toolchain, emulator, and even the CPU (based on OpenRISC) have
        been released to GitHub. Among other things, I was solely
        responsible for implementing the real-time multitasking
        operating system for the project.\\
%        
%        Earlier in the project I was responsible for creating the initial software that used a fake FAT filesystem as a means of socket communication between an external device and an Android device without need for a kernel driver.\\

        Launch video (Google I/O 2015): \hfill \href{http://goo.gl/5mZrVR}{\textbf{http://goo.gl/5mZrVR}}\\
        Source code: \hfill \href{http://goo.gl/0pbsk7}{\textbf{http://goo.gl/0pbsk7}}
    }
    
  \resumeentry
      {2013}
      {
        \vspace{0.20cm}
        \begin{tikzpicture}%
          \node[circle, inner sep=0.6cm, fill overzoom image=images/darpa.jpg] () {};%
        \end{tikzpicture}
      }
      {X86 Hypervisor and CPU Instruction fuzzer}
      {Inverse Limit (for DARPA)}
      {

        The MAIM project (Micro-architecture Instruction Mining) was
        one of three Cyber Fast Track proposals that DARPA accepted
        from Inverse Limit. It consisted of an x86 instruction fuzzer
        and a cross-platform hypervisor to execute the instructions
        and compare their behavior on different implementations. The
        project identified several undocumented differences between
        Intel, AMD and VIA architectures.\\

        Whitepaper: \hfill \href{http://goo.gl/Kwa3Rf}{\textbf{http://goo.gl/Kwa3Rf}}\\
      }
      
  \resumeentry
      {2010-2011}
      {
        \vspace{0.15cm}
        \begin{tikzpicture}%
          \node[inner sep=0.85cm,fill overzoom image=images/bpointsys.jpg] () {};%
        \end{tikzpicture}
      }
      {Complete TCP/IP Stack for Attack Traffic}
      {BreakingPoint Systems}
      {

        I redesigned BreakingPoint's existing network security test
        framework from a traffic simulator into a fully featured
        TCP/IP stack that could test live applications while
        transparently applying any number of advanced network evasion
        techniques. At the same time, by integrating concurrency into
        the new design and strategically replacing components with C
        extensions, the performance was quadrupled.}

\newpage
      
\resumesection{Work History}

    
  \resumeentry
    {2016-present}
    {
      \vspace{0.72cm}
      \begin{tikzpicture}%
        \node[inner sep=1.05cm,fill overzoom image=images/playstudios.png] () {};%
      \end{tikzpicture}        
    }
    {PlayStudios}
    {Software Engineer}
    {

      About half (at the time) of the state lotteries in the US were
      managed by GTECH. As a Systems Administrator at their Austin
      datacenter, I was directly responsible for the ultra-high
      availability lottery systems of Texas, California, Idaho,
      Kansas, Jamaica, and Washington.
      
      \begin{resumeitemize}
      \item{Maintained and operated high volume lottery servers that handle thousands of transactions per minute in an environment where large government fines are imposed for downtime}
      \item{Administrated OpenVMS, AIX, Linux, Tru64  Unix, Windows (98, 2000, XP, 2003), MSSQL, and Sybase}
      \item{Automated a large part of the Operations department using Perl, Shell (csh and dcl), and BMC Control-M}
      \end{resumeitemize}
    }

  \resumeentry
      {2014-2016}
      {
        \vspace{0.72cm}
        \begin{tikzpicture}%
          \node[inner sep=1.3cm,fill overzoom image=images/orthokinematics.png] () {};%
        \end{tikzpicture}        
      }
    {Ortho Kinematics}
    {Software Engineer}
    {

      BreakingPoint's product is designed to be an ultra-high
      performance tool for testing network devices. It generates
      realistic network traffic at 100+ gigabits per second while
      monitoring the device under test for reporting.
      
      \begin{resumeitemize}
      \item{Complete TCP/IP implementation in Ruby (see above)}
      \item{Implemented framework for network traffic simulation}
      \item{Discovered and reported new 3rd-party vulnerabilities}
      \item{Performed differential patch analysis on Microsoft updates monthly}
      \item{Maintained product coverage of important security vulnerabilities} 
      \item{Sample blog posts: \href{http://goo.gl/8yzJFv}{\textbf{http://goo.gl/8yzJFv}} - \href{http://goo.gl/GnWZGX}{\textbf{http://goo.gl/GnWZGX}}  }
      \end{resumeitemize}

    }

   \resumeentry
      {2010-2014}
      {
        \vspace{1.20cm}
        \begin{tikzpicture}%
          \node[inner sep=1.35cm,fill overzoom image=images/auctiva.png] () {};%
        \end{tikzpicture}
      }
    {Auctiva Corperation}
    {Developer}
    {

      Leviathan is a small consultancy focusing on challenging niche
      projects. I was brought on to help facilitate long-term R\&D
      projects and to assist the consulting group as needed.
      
      \resumeentryheading{Mayor Myer (DARPA Research Program)}
      \begin{resumeitemize}
      \item{Designed x64 polymorphic shellcode encoder for Metasploit}
      \item{GNU libc heap corruption detector}
      \item{Maintained and extended x86 JIT compiler and emulator}
      \end{resumeitemize}
      \resumeentryheading{Consulting}
      \begin{resumeitemize}
      \item{Audited Intel ME firmware}
      \item{Developed fuzzer for Intel ME applications}
      \item{Audits of embedded Java applications}
      \item{Android Research}
      \end{resumeitemize}
      
    }
    
\end{resume}
\end{document}
