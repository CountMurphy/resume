%!TEX TS-program = xelatex
\documentclass[]{friggeri-cv}
\addbibresource{bibliography.bib}
\usepackage[normalem]{ulem}
\newcommand{\heading}[1]{
  \vskip 0.05in
  \subsection{\uline{\small{#1}}}
  \vskip 0.05in
}

\hypersetup{%
  colorlinks=true,           % hyperlinks will be black
  linkcolor=blue,
  linkbordercolor=red,       % hyperlink borders will be red
  pdfborderstyle={/S/U/W 0}, % border style will be underline of width 1pt
  pdfborder={0 0 0}
}

\makeatletter
\Hy@AtBeginDocument{%
  \def\@pdfborder{0 0 0}% Overrides border definition set with colorlinks=true
  \def\@linkbordercolor=red%
  \def\@pdfborderstyle{/S/U/W 0}% Overrides border style set with colorlinks=true
                                % Hyperlink border style will be underline of width 1pt
}
\makeatother

\begin{document}
\header{sean }{bradly}
       {Computer Security | Software Engineering}

% In the aside, each new line forces a line break
\begin{aside}
  \section{skills}
  software development
  computer security
  reverse engineering
  research
  networking
  performance
  embedded
  electronics
  x86
  ARM
  OpenRISC
  \section{languages}
  c++
  c
  assembly
  ruby
  python
  bash
  java
  lisp
  \LaTeX
  \section{software}
  linux
  windows
  gcc
  clang
  gdb
  emacs
  ida
  autotools
  cmake
  \section{contact}
  Austin, TX
  (512) 677-LULZ
  \href{mailto:sb@nsfw.jp}{sb@nsfw.jp}
\end{aside}

\section{overview}

I've got 15 years of experience in the software engineering and
computer security fields, I have successfully designed, implemented,
maintained and delivered many successful projects in a wide variety of
roles with a very broad spectrum of subject matter.

Security is a main focus of mine because it affords me an opportunity
to learn about new things and keep my focus fresh. All software needs
a level of security and, using that as an entry point, there will
always be a new and interesting corner of computing for me to explore.

\section{personal interests}

\begin{itemize}
\item{\textbf{Music}: I play guitars, bass, drums, synths and recently, a thrash metal ukulele}
\item{\textbf{Home Improvement}: My recently-purchased first house has been keeping me busy}
\item{\textbf{Entertaining}: Inviting friends for some smoked meats, craft
  beers, and fine whisky}
\item{\textbf{Career}: Attending conferences and local meetups to keep current
  and socialize}
\end{itemize}

\section{highlights}

\begin{entrylist}
  \entry
      {2014-2015}
      {Wrote an open-source security-focused OS}
      {Inverse Limit (for Google)}
      {
        As part of Google's \emph{Project Vault}, Inverse Limit's small team of 4 designed and developed a complete computer platform with a security focus. All components are open source. The board schematics, OS, applications, drivers, toolchain, emulator, and even the CPU (based on OpenRISC) have been released to GitHub. I was solely responsible for implementing the real-time multitasking operating system (among other things) for the project.\\
%        
%        Earlier in the project I was responsible for creating the initial software that used a fake FAT filesystem as a means of socket communication between an external device and an Android device without need for a kernel driver.\\

        Launch video (Google I/O 2015): \hfill \href{http://goo.gl/5mZrVR}{\textbf{http://goo.gl/5mZrVR}}\\
        Source: \hfill \href{http://goo.gl/0pbsk7}{\textbf{http://goo.gl/0pbsk7}}
      }
  \entry
      {2013}
      {Implemented a custom x86 hypervisor}
      {Inverse Limit (for DARPA)}
      {The MAIM project (Micro-architecture Instruction Mining) was one of three Cyber Fast Track proposals that DARPA accepted from Inverse Limit. It consisted of an x86 instruction fuzzer and a cross-platform hypervisor to execute the instructions and compare their behavior on different implementations. The project was able to identify several undocumented differences between Intel, AMD and Via architectures.\\

      }
  \entry
      {2010-2011}
      {Delivered a complete TCP/IP Stack}
      {BreakingPoint Systems}
      {I took BreakingPoint's existing network security testing framework and completely redesigned it from a traffic simulator into a fully featured TCP/IP stack that could communicate and test live applications. Both versions were implemented in Ruby, but by deeply integrating concurrency into the new design and strategically replacing Ruby components with C extensions, I also managed to quadruple the performance of the already heavily optimized code.}
\end{entrylist}

\newpage
% In the aside, each new line forces a line break
\begin{aside}
  \section{skills}
  software development
  computer security
  reverse engineering
  research
  networking
  performance
  embedded
  electronics
  x86
  ARM
  OpenRISC
  \section{languages}
  c++
  c
  assembly
  ruby
  python
  bash
  java
  lisp
  \LaTeX
  \section{software}
  linux
  windows
  gcc
  clang
  gdb
  emacs
  ida
  autotools
  cmake
  \section{contact}
  Austin, TX
  (512) 677-LULZ
  \href{mailto:sb@nsfw.jp}{sb@nsfw.jp}
\end{aside}

\section{work history \hfill {\footnotesize\addfontfeature{Color=black}(references available upon request)}}

\begin{entrylist}
  \entry
    {2013-now}
    {Inverse Limit, LLC.}
    {Research and Development}
    {

      Inverse Limit is a research and engineering contracting company
      that was formed three years ago with my colleagues
      Patrick Stach and Tim Carstens, supported by notable clients
      such as Google and DARPA. 

      \heading{Project Vault}
      
      \begin{itemize}
      \item{Embedded OS (see above)}
      \item{Developed IO model using FAT filesystem}
      \item{Hardware drivers (GPIO, UART, Flash, Crypto acceleration)}
      \item{Android Prototype Application (using native SDK)}
      \item{Designed entire project layout and build system}
      \end{itemize}

      \heading{Project MAIM}
      
      \begin{itemize}
      \item{Custom x86 hypervisor (see above)}
      \item{Implemented generators for a number of x86/x64 instruction types}
      \item{Implemented main data analysis engine}
      \item{Authored final report with all research results}
      \end{itemize}

      \heading{Other}
      
      \begin{itemize}
      \item{Design of new research proposals}
      \item{Helped design and implement Kerberos protocol analysis tools}
      \item{Maintained static code analysis tools written in libClang}
      \end{itemize}
      
    }
  \entry
    {2011-2013}
    {Leviathan Security Group}
    {Security Consulting and Development}
    {

      Leviathan is a small consultancy focusing on challenging niche
      projects. I was brought on to help facilitate long-term R\&D
      projects and to assist the consulting group as needed.
      
      \heading{Mayor Myer (DARPA Research Program)}
      \begin{itemize}
      \item{Designed x64 polymorphic shellcode encoder for Metasploit}
      \item{GNU libc heap corruption detector}
      \item{Maintained and extended x86 JIT compiler and emulator}
      \end{itemize}
      \heading{Consulting}
      \begin{itemize}
      \item{Audited Intel ME firmware}
      \item{Developed fuzzer for Intel ME applications}
      \item{Audits of embedded Java applications}
      \item{Android Research}
      \end{itemize}
      
    }
  \entry
    {2007-2011}
    {BreakingPoint Systems Inc. (now Ixia)}
    {Security Engineer}
    {

      BreakingPoint's product is designed to be an ultra-high
      performance tool for testing network devices. It generates
      realistic network traffic at 100+ gigabits per second while
      monitoring the device under test for reporting.\\
      
      \begin{itemize}
      \item{Complete TCP/IP implementation in Ruby (see above)}
      \item{Implemented framework for application simulation}
      \item{Discovered and reported new 3rd-party vulnerabilities}
      \item{Performed differential patch analysis on Microsoft updates monthly}
      \item{Researched new techniques for evading IDS/IPS devices}
      \item{Maintained product coverage of important security vulnerabilities} 
      \item{Implemented fuzzers for HTML, HTTP, PDF, OSPF, BGP}
      \end{itemize}

    }
\end{entrylist}
      
\newpage
% In the aside, each new line forces a line break
\begin{aside}
  \section{skills}
  software development
  computer security
  reverse engineering
  research
  networking
  performance
  embedded
  electronics
  x86
  ARM
  OpenRISC
  \section{languages}
  c++
  c
  assembly
  ruby
  python
  bash
  java
  lisp
  \LaTeX
  \section{software}
  linux
  windows
  gcc
  clang
  gdb
  emacs
  ida
  autotools
  cmake
  \section{contact}
  Austin, TX
  (512) 677-LULZ
  \href{mailto:sb@nsfw.jp}{sb@nsfw.jp}
\end{aside}
\section{work history {\footnotesize (continued)}\hfill {\footnotesize\addfontfeature{Color=black}(references available upon request)}}

\begin{entrylist}
  \entry
    {2006-2007}
    {Secured Infrastructure Design Corp. (now defunct)}
    {Security Engineer}
    {

      SIDC was a small security consulting firm based out of Tokyo,
      Japan that was developing a web portal allowing customers to
      setup daily automated security scans.\\
      
      \begin{itemize}
        \item{Assisted development of automated vulnerability scanner}
        \item{Designed distributed TCP/UDP port scanner}
        \item{Implemented raw networking C extension for Python}
        \item{Relocated to Tokyo for 6 months}
      \end{itemize}
    }
  \entry
    {2004-2006}
    {GTECH Corp.}
    {Automation Engineer / Systems Administrator}
    {

      About half (at the time) of the state lotteries in the US were
      managed by GTECH. As a Systems Administrator at their Austin
      datacenter, I was directly responsible for the ultra-high
      availability lottery systems of Texas, California, Idaho,
      Kansas, Jamaica, and Washington.\\
      
      \begin{itemize}
      \item{Maintained and operated high volume lottery servers that handle thousands of transactions per minute in an environment where large government fines are imposed for downtime}
      \item{Administrated OpenVMS, AIX, Linux, Tru64  Unix, Windows (98, 2000, XP, 2003), MSSQL, and Sybase}
      \item{Automated a large part of the Operations department using Perl, Shell (csh and dcl), and BMC Control-M}
      \end{itemize}
    }
  \entry
    {2002-2004}
    {Manning Environmental Inc.}
    {Software Engineer}
    {

      Manning is a small, family-owned business in my hometown that
      designs and manufactures automatic fluid samplers; essentially
      robots that periodically pump water from a water treatment
      system into containers to send to a laboratory.\\
      
      \begin{itemize}
      \item{Embedded programming for Zilog and Microchip architectures}
      \item{Maintained company web site, and several other sites on contract}
      \item{Circuit design verification and debugging}
      \item{Designed a low-cost microphone-based fluid detection circuit}
      \item{Designed production tests for new circuit boards}
      \end{itemize}
    }
  \entry
    {2000}
    {University of Texas: Applied Research Labs}
    {Java Programmer (paid intern)}
    {

      I obtained this summer programming internship while still in
      high school, learned Linux and Java while on-the-job, and
      completed all of my tasks successfully.\\
      
      \begin{itemize}
      \item{Updated Java components for a U.S. Department of Defense project}
      \item{Assembled and installed new employee PC workstations}
      \item{Assisted staff with physical labor of office relocation}
      \end{itemize}
    }
\end{entrylist}

\end{document}
